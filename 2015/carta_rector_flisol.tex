\documentclass[xcolor, x11names, letterpaper, 12pt]{letter}
\usepackage[table]{xcolor}
\definecolor{maroon}{cmyk}{0,0.87,0.68,0.32}


\usepackage[spanish]{babel}
\usepackage[utf8]{inputenc}

\usepackage[letterpaper]{geometry}
\usepackage{latexsym,amsmath,amssymb,amsthm}
\usepackage{graphicx}
\usepackage{pifont}
\usepackage[pdftex=true,colorlinks=true,plainpages=false]{hyperref}
\hypersetup{urlcolor=blue}
\hypersetup{linkcolor=black}
\hypersetup{citecolor=black}
\usepackage{lastpage}
\usepackage{url}
\usepackage{anysize}
\marginsize{2.2cm}{2.2cm}{0.8cm}{2cm}
\usepackage{fancyhdr}
\usepackage{anyfontsize}
\usepackage{tocbibind}
\usepackage{eso-pic}
\usepackage{mathptmx}
\usepackage{draftwatermark}
\usepackage{multirow}
%\usepackage[usenames,dvipsnames]{xcolor}
\usepackage{blindtext}
\usepackage{longtable}
%\usepackage{showframe}
%-------------------------

\fancypagestyle{plain}{
	\lhead{ }
	\chead{
		{ \tt \Huge Universidad Mayor de San Simón }\\
		{ \tt \huge Facultad de Ciencias y Tecnología }\\
		{ \tt \large Sociedad Científica de Estudiantes de Sistemas e Informática }
	}
	\rhead{ }
	
	\lfoot{
		{ \scriptsize \tt \url{http://www.scesi.org} } \\
		{ \scriptsize \tt \url{http://www.scesi.memi.umss.edu.bo} }
	}
	\cfoot{
		{ \scriptsize \tt Prolongación calle Sucre - Parque la Torre - Teléfono: 74312946	 } \\
		{ \scriptsize \tt Bloque central de la FCyT - ultimo piso } \\
	}
	\rfoot{
		{ \small \tt root@scesi.org } \\
	 	\thepage/\pageref{LastPage} 
	}
}


\renewcommand{\headrule}{
	{
		\color{brown}
		\hrule width\headwidth height\headrulewidth\vskip-\headrulewidth
	}
}
\renewcommand{\footrule}{
	{
		\color{brown}%
		\vskip-\footruleskip\vskip-\footrulewidth
		\hrule width\headwidth height\footrulewidth\vskip\footruleskip
	}
}
\renewcommand{\headrulewidth}{3pt}
\renewcommand{\footrulewidth}{3pt}
\headheight = 56pt
\headsep = 10pt
\footskip = 28pt

\pagestyle{plain}
\makeatletter
\let\ps@empty\ps@plain
\let\ps@firstpage\ps@plain
\makeatother

%--------------------------

\newcommand\BackgroundPic{
	\put(502,724){
		\parbox[b][\paperheight]{\paperwidth}{
			\includegraphics[scale=0.9]{../../img/logoSinFondo.png}
		}
	}
	\put(40,724){
		\parbox[b][\paperheight]{\paperwidth}{
			\includegraphics[scale=0.08]{../../img/logoUMSS.jpg}
		}
	}
}

\setlength{\parindent}{0mm}
\setlength{\parskip}{3mm}

%---------------------------
\SetWatermarkAngle{0}
%\SetWatermarkLightness{0.9}
%\SetWatermarkFontSize{1cm}
\SetWatermarkScale{3}
\SetWatermarkText{\includegraphics[scale=0.4]{../../img/waterMark2.png}}
%--------------------------------------

\begin{document}
\AddToShipoutPicture{\BackgroundPic}
	%	\tableofcontents



\date{Cochabamba, \today}
\begin{letter}{
    Señor: \\
    Mgr. Waldo Jimenez Valdivia\\
    \textbf{Rector Universidad Mayor de San Sim\'on - UMSS}\\
    \underline {Presente}. -
}

\begin{center}
    \opening{\textbf{ Ref.:  Solicitud de apoyo a la Flisol 2015}}
\end{center}

Estimado Rector:

A tiempo de saludarlo cordialmente, me dirijo a usted con la finalidad 
de pedir su apoyo para llevar a cabo la Festival Latinoamericano de 
Instalaci\'on de Software Libre FLISoL 2015, destinado a difundir, 
compartir el uso del Software Libre que en esta ocasion es organizada 
por la Sociedad Cient\'ifica de Estudiantes de Sistemas Inform\'atica 
- SCESI, en ambientes del Centro MEMI el d\'ia s\'abado 25 de abril 
donde se iniciara a las 9:00 AM y finalizando a las 5:00 PM.

El apoyo consiste en proporcionar materiales de escritorio como:
\begin{center}
\begin{enumerate}
   \item[CD]  Destinado para distribuir programas compatibles con Windows y Mac.
   \item[DVD] Destinado para distribuir las distribuciones linux (Debian, Ubuntu, Fedora, ...)
   \item[Hojas] Destinado para imprimir afiches del evento.
\end{enumerate}
\end{center}
%Con el consiguiente apoyo se pretende impulsar el uso del Software 
%Libre tanto en el a\'mbito educativo y profesional, con esto se estar\'ia 
%cumpliendo las regulaciones establecidas por los decretos 
%1793 del 13 de noviembre del 2013 y 1874 del 23 de enero del 2014 que 
%mencionan el apoyo y el uso de software libre a instituciones publicas.


La Universidad Mayor de San Sim\'on, siempre se a destacado por su calidad en 
la ense\~nanza y su apoyo a la formaci\'on profesional a lo largo de los a\'nos. 
La SCESI a travez de este tipo de eventos pretende reafirmar la meta para con 
la UMSS , ademas de cumplir con los objetivos propios del software libre y de 
esta forma colaborar con el avance de la tecnolog\'ia en nuestro pais.


Sin otro particular y a la espera de una respuesta favorable a esta solicitud, 
me despido muy cordialmente.

\vspace{1.4cm}

\begin{center}
................................\\
Ubaldino Zurita \\
Presidente SCESI-UMSS \\
\end{center}
\end{letter}\textsc{}
\end{document}
