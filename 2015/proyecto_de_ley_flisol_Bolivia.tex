\documentclass[11pt,letterpaper]{article}

\usepackage[letterpaper]{geometry}

\usepackage[spanish]{babel}
\usepackage[utf8]{inputenc}
\usepackage{latexsym,amsmath,amssymb,amsthm}
\usepackage{graphicx}
\usepackage{pifont}
\usepackage[pdftex=true,colorlinks=true,plainpages=false]{hyperref}
\hypersetup{urlcolor=blue}
\hypersetup{linkcolor=black}
\hypersetup{citecolor=black}
\usepackage{lastpage}
\usepackage{url}
\usepackage{anysize}
\marginsize{2.2cm}{2.2cm}{0.8cm}{1.7cm}
\usepackage{fancyhdr}
\usepackage{anyfontsize}
\usepackage{tocbibind}
\usepackage{eso-pic}
\usepackage{mathptmx}
\usepackage{draftwatermark}
\usepackage{multirow}
%\usepackage[usenames,dvipsnames]{xcolor}
\usepackage{blindtext}
\usepackage{longtable}
\usepackage[document]{ragged2e}


%-------------------------

\pagestyle{fancy}
\lhead{ }
\chead{
	{ \Huge \tt Universidad Mayor de San Simón }\\
	{ \tt \huge Facultad de Ciencias y Tecnología }\\
	{ \tt \large Sociedad Científica de Estudiantes de Sistemas e Informática }
}
\rhead{ }

\lfoot{
	{ \scriptsize \tt \url{http://www.scesi.org} } \\
	{ \scriptsize \tt \url{http://www.scesi.memi.umss.edu.bo} }
}
\cfoot{
	{ \scriptsize \tt Prolongación calle Sucre - Parque la Torre - Teléfono: 74312946 } \\
	{ \scriptsize \tt Bloque central de la FCyT - ultimo piso } \\
}
\rfoot{
	{ \small \tt root@scesi.org } \\
 	\thepage/\pageref{LastPage} 
}

\renewcommand{\headrule}{
	{
		\color{brown}
		\hrule width\headwidth height\headrulewidth\vskip-\headrulewidth
	}
}
\renewcommand{\footrule}{
	{
		\color{brown}%
		\vskip-\footruleskip\vskip-\footrulewidth
		\hrule width\headwidth height\footrulewidth\vskip\footruleskip
	}
}
\renewcommand{\headrulewidth}{3pt}
\renewcommand{\footrulewidth}{3pt}
\headheight = 56pt
\headsep = 10pt
\footskip = 28pt
% %\oddsidemargin = 120pt
%\textwidth = 300pt
%--------------------------

\newcommand\BackgroundPic{
	\put(502,728){
		\parbox[b][\paperheight]{\paperwidth}{
			\includegraphics[scale=0.9]{../../img/logoSinFondo.png}
		}
	}
	\put(40,729){
		\parbox[b][\paperheight]{\paperwidth}{
			\includegraphics[scale=0.08]{../../img/logoUMSS.jpg}
		}
	}
}

\setlength{\parindent}{0mm}
\setlength{\parskip}{3mm}

%---------------------------
\SetWatermarkAngle{0}
%\SetWatermarkLightness{0.9}
%\SetWatermarkFontSize{1cm}
\SetWatermarkScale{3}
\SetWatermarkText{\includegraphics[scale=0.4]{../../img/waterMark2.png}}
%--------------------------------------



\begin{document}
\AddToShipoutPicture{\BackgroundPic}
	%	\tableofcontents

~\\

\begin{center}
{\tt \Large PROYECTO DE LEY\\

Que declara al Festival Latinoamericano de Instalación de Software Libre (FLISOL) - Bolivia como actividad oficial del Estado Plurinacional de Bolivia\\

LA ASAMBLEA LEGISLATIVA PLURINACIONAL\\

D E C R E T A:}
\end{center}
{\tt
\begin{description}

\item[Artículo 1.-]~\\
(Objeto) Se declara al Festival Latinoamericano de Instalación de Software Libre (FLISOL) - Bolivia como actividad oficial del Estado Plurinacional de Bolivia, debiendo desarrollarse a nivel nacional, con la finalidad de promover el uso del Software Libre, mediante la información a los y las bolivianas, de su filosofía, alcances, avances y desarrollo.

\item[Artículo 2.-]~\\
(Ejecución) Los Ministerios y sus Entidades Descentralizadas a nivel nacional, las Entidades Territoriales Autónomas del país, anualmente y durante la última semana del mes de abril, en ocasión de la realización del Festival Latinoamericano de Instalación de Software Libre (FLISOL) - Bolivia, deberán, a través de sus unidades y/o dependencias de Tecnologías de la Información y Comunicación (TIC), organizar seminarios, conferencias, charlas, foros, publicaciones impresas y digitales, concursos y toda otra actividad, para todos sus servidores públicos y población en general, a objeto de enseñar, concientizar, instalar y promover el uso y aplicación de software libre.

\item[Artículo 3.-]~\\
\begin{itemize}
\item[I.] (Cumplimiento)  El Ministerio de la Presidencia, en combinación con las comunidades y activistas del Software Libre, queda encargado de coordinar con todos los niveles del Estado Plurinacional, el cumplimiento de la presente Ley.
\item[II.] El Ministerio de la Presidencia, en un plazo no mayor a 60 días posteriores a la conclusión de la FLISOL- Bolivia, deberá recolectar, evaluar y publicar un informe impreso y digital del cumplimiento de la presente Ley.
\end{itemize}
\end{description}
Remítase al Órgano Ejecutivo, para fines constitucionales.\\

\begin{center}
Sen. Nélida Sifuentes Cueto\\
1ra. Vicepresidenta Cámara de Senadores \\
PROYECTISTA \\
\end{center}
}
{\it Fuente:}\\
 \url{ http://softwarelibrebolivia.blogspot.com/2015/03/buscan-declarar-al-festival.html }
\end{document}
