\documentclass[xcolor, x11names, letterpaper, 12pt]{letter}
\usepackage[table]{xcolor}
\definecolor{maroon}{cmyk}{0,0.87,0.68,0.32}


\usepackage[spanish]{babel}
\usepackage[utf8]{inputenc}

\usepackage[letterpaper]{geometry}
\usepackage{latexsym,amsmath,amssymb,amsthm}
\usepackage{graphicx}
\usepackage{pifont}
\usepackage[pdftex=true,colorlinks=true,plainpages=false]{hyperref}
\hypersetup{urlcolor=blue}
\hypersetup{linkcolor=black}
\hypersetup{citecolor=black}
\usepackage{lastpage}
\usepackage{url}
\usepackage{anysize}
\marginsize{2.2cm}{2.2cm}{0.8cm}{2cm}
\usepackage{fancyhdr}
\usepackage{anyfontsize}
\usepackage{tocbibind}
\usepackage{eso-pic}
\usepackage{mathptmx}
\usepackage{draftwatermark}
\usepackage{multirow}
%\usepackage[usenames,dvipsnames]{xcolor}
\usepackage{blindtext}
\usepackage{longtable}
%\usepackage{showframe}
%-------------------------

\fancypagestyle{plain}{
	\lhead{ }
	\chead{
		{ \tt \Huge Universidad Mayor de San Simón }\\
		{ \tt \huge Facultad de Ciencias y Tecnología }\\
		{ \tt \large Sociedad Científica de Estudiantes de Sistemas e Informática }
	}
	\rhead{ }
	
	\lfoot{
		{ \scriptsize \tt \url{http://www.scesi.org} } \\
		{ \scriptsize \tt \url{http://www.scesi.memi.umss.edu.bo} }
	}
	\cfoot{
		{ \scriptsize \tt Prolongación calle Sucre - Parque la Torre - Teléfono: 74312946	 } \\
		{ \scriptsize \tt Bloque central de la FCyT - ultimo piso } \\
	}
	\rfoot{
		{ \small \tt root@scesi.org } \\
	 	\thepage/\pageref{LastPage} 
	}
}


\renewcommand{\headrule}{
	{
		\color{brown}
		\hrule width\headwidth height\headrulewidth\vskip-\headrulewidth
	}
}
\renewcommand{\footrule}{
	{
		\color{brown}%
		\vskip-\footruleskip\vskip-\footrulewidth
		\hrule width\headwidth height\footrulewidth\vskip\footruleskip
	}
}
\renewcommand{\headrulewidth}{3pt}
\renewcommand{\footrulewidth}{3pt}
\headheight = 56pt
\headsep = 10pt
\footskip = 28pt

\pagestyle{plain}
\makeatletter
\let\ps@empty\ps@plain
\let\ps@firstpage\ps@plain
\makeatother

%--------------------------

\newcommand\BackgroundPic{
	\put(502,724){
		\parbox[b][\paperheight]{\paperwidth}{
			\includegraphics[scale=0.9]{../../img/logoSinFondo.png}
		}
	}
	\put(40,724){
		\parbox[b][\paperheight]{\paperwidth}{
			\includegraphics[scale=0.08]{../../img/logoUMSS.jpg}
		}
	}
}

\setlength{\parindent}{0mm}
\setlength{\parskip}{3mm}

%---------------------------
\SetWatermarkAngle{0}
%\SetWatermarkLightness{0.9}
%\SetWatermarkFontSize{1cm}
\SetWatermarkScale{3}
\SetWatermarkText{\includegraphics[scale=0.4]{../../img/waterMark2.png}}
%--------------------------------------

\begin{document}
\AddToShipoutPicture{\BackgroundPic}
	%	\tableofcontents



\date{Cochabamba, \today}
\begin{letter}{
    Señor: \\
    Mgr. Lic. Jorge Mario Ponce Coca \\
    \textbf{Director Departamental de Educación}\\
    \underline {Presente}. -
}

\begin{center}
    \opening{\textbf{Ref.:  Solicitud de espacios para la difusión del Festival Latinoamericano de Instalación de Software Libre (FLISoL)}}
\end{center}

Estimado Mgr. Lic. Jorge Mario Ponce Coca,


A tiempo de saludarlo y desearle éxito en las funciones que desempeña, la Sociedad Científica de Sistemas e Informática (SCESI), perteneciente a la Universidad Mayor de San Simón (UMSS), encargado de la construcción de herramientas destinadas a la Olimpiada Boliviana de Informática (OBI).\\

Le informamos que actualmente la SCESI organiza y coordina el Festival Latinoamericano de Instalación de Software Libre (FLISoL) en Cochabamba con {\bf sede en la UMSS en los ambientes del centro MEMI dentro la Facultad de Ciencias y Tecnología (FCyT), mismo que iniciará a horas 9:00 y finalizará a horas 17:00 el día sábado 25 de Abril del presente año}.\\


Para esta actividad hemos considerado contar con su apoyo, facilitándonos una autorización para la difusión en los colegios del cercado, informando acerca del evento para que los estudiantes y maestros puedan asistir a la {\bf FLISoL Cochabamba 2015}.\\


Adjuntamos un \textbf{anexo} en el cual se explica los detalles concernientes al evento \textbf{FLISoL}.

Sin otro particular y a la espera de una respuesta favorable a esta solicitud, me despido de usted.

Atentamente, 

\vspace{1.4cm}

\begin{center}
................................\\
Ubaldino Zurita \\
Presidente SCESI-UMSS \\
\end{center}
\end{letter}\textsc{}

\begin{center}
{\Huge Anexo}
\end{center}

\textbf{Antecedentes FLISoL:}\\

El FLISoL es el evento de difusión de Software Libre más grande en Latinoamérica y está dirigido a todo tipo de público: estudiantes, académicos, empresarios, trabajadores, funcionarios públicos, entusiastas y aun personas que no poseen mucho conocimiento informático.\\

La \textbf{entrada es gratuita} y su principal objetivo es \textbf{promover el uso del software libre}, dando a conocer al público en general su filosofía, alcances, avances y desarrollo.
El evento es organizado por las diversas comunidades locales de Software Libre y se desarrolla simultáneamente con eventos en los que se instala, de manera gratuita y totalmente legal, software libre en las computadoras que llevan los asistentes. Además, en forma paralela, se ofrecen charlas, ponencias y talleres, sobre temáticas locales, nacionales y latinoamericanas en torno al Software Libre, en toda su gama de expresiones: artística, académica, empresarial y social.\\

\textbf{Objetivos del evento:}\\

El principal objetivito es promover el uso del software libre a los estudiantes y dar talleres a los profesores.

\textbf{Coordinación con SEDUCA:}\\
Mediante su persona solicitamos:\\

Nos colabore en el envió de una circulante a colegios para la capacitación de maestros en el área de ofimática, manejo de GNU/Linux y programas de computadora para las olimpiadas bolivianas de informática, la capacitación no tiene ningún costo y se dará un certificado a los maestros que asistan.\\

Esperamos coordinar con su persona o institución para que los certificados tengan un valor curricular en las competencias de méritos que evalúa la dirección departamental de educación si fuera viable.\\

Le pedimos una autorización de 10 minutos en los colegios para dar a conocer lo que se realizara en FLISoL.\\
Autorización para colocar afiches en los colegios para el evento FLISoL.\\
\newpage
\textbf{Talleres:}\\
Los talleres serán realizados por personas reconocidas en el área de ofimática, instructores de las olimpiadas bolivianas de informatica y robotica como también personas que  trabajan  currículas de educación en base al software libre:\\

\centering
\rowcolors{1}{ }{maroon!10}
\begin{tabular}{lll} \hline
\rowcolor{maroon!40} $Nro$ & $ Taller $ & $Expositor$\\\hline
1 & Aprendiendo a escribir documentos con LaTeX & Benjamin Perez\\
2 & Colaboracion en tiempo real con TogetherJS & Diego Landa\\
3 & Control de motores con arduino & Daniel Saguez\\
4 & Creando aplicaciones moviles hibridas con tecnologias web & Lipa Challapa Jorge Rubens\\
5 & Como ser parte de una comunidad, Fedora te invita & Gonzalo Nina Mamani\\
6 & Herramientas libres para recolección de información & Gonzalo Nina Mamani\\
7 & Desarrolladores con Código Abierto & Oscar Rolando Gamboa Acho\\
8 & Pleni Sistema de recoleccionde informacion & Cristhian Lima Saravia\\
9 & Software libre en la empresa & Rodolfo Negron Manchego \\
10 & UrbanTerror FPSLibre & Rodolfo Negron Manchego \\\hline
\end{tabular}


\end{document}
